%Makros für IDs


\chapter{Nichtfunktionale Anforderungen}
    \SetLabelAlign{fixedwidth}{\hss\llap{\makebox[2.5em][l]{#1}}}
    \begin{enumerate}[font={\bfseries},label={NA\arabic*}0, wide=0pt, labelindent=1em, leftmargin=*]
        
        \item Schneller Start \newline
        Der Workflow Editor soll bei einer durchschnittlichen Internetverbindung nach maximal 5 Sekunden im Internetbrowser geladen werden.
        
        \item Robustheit des Workflow-Editors \newline
        Da alle grafische Elemente leichtgewichtig sind arbeitet der Workflow Editor ohne Verzögerungen.
        
        \item Anpassbarkeit \newline
        Der Workflow Editor und alle dazugehörige Tools sollen bei einer Auflösung von 1280 x 1024 Pixel ohne Scrollbars sichtbar sein.
        
        \item Benutzerfreundlichkeit\newline
        Das Produkt soll verständlich und intuitiv sein, damit auch Personen, die keine IT-Experten sind, ohne komplizierte Anleitung, Workflows erstellen und bearbeiten können.
        
        \item Sortieren der Workflow-Liste\newline
        Die Liste aller in der Datenbank gespeicherten Workflows kann nach Workflow-Namen, Erstellungsdatum, Ausführungsstatus, Besitzer oder nach Kombinationen dieser Filter sortiert werden.
        
        \item Zuverlässigkeit\newline
        Das Workflow-System darf maximal sieben Tage pro Jahr nicht verfügbar sein.
        
        \item Anzahl der Workflows\newline
        Um Serverüberlastungen zu verhindern ist die Anzahl der gleichzeitig ausführbaren Workflows beschränkt. Die Zuverlässigkeit und Korrektheit der Ausführung wird bei maximal 100 Workflows garantiert.
        
        \item Änderbarkeit bzw. Erweiterbarkeit\newline
        Die Benutzeroberfläche ist leicht austauschbar.
        
        \item Serverbelastung\newline
        Die Serverantwort soll im Schnitt nicht länger als 3 Sekunden dauern.
        Der Server kann bis zu 1000 verschiedene Benutzer gleichzeitig bedienen.
        
    \end{enumerate}